\documentclass[11pt]{amsart}
\usepackage{setspace,url,hyperref}
\usepackage{amssymb,stmaryrd}
\usepackage{proof}
\usepackage[usenames,dvipsnames,svgnames,table]{xcolor}
\newtheorem{thm}{Theorem}[section]
\newtheorem{prop}[thm]{Proposition}
\newtheorem{lem}[thm]{Lemma}
\newtheorem{cor}[thm]{Corollary}
\theoremstyle{definition}
\newtheorem{definition}[thm]{Definition}
\newtheorem{example}[thm]{Example}
\theoremstyle{remark}
\newtheorem{remark}[thm]{Remark}
\numberwithin{equation}{section}

\def\InputModeColorName{MidnightBlue}
\def\OutputModeColorName{Maroon}
\newcommand\IMode[1]{{\color{\InputModeColorName}{#1}}}
\newcommand\OMode[1]{{\color{\OutputModeColorName}{#1}}}

\newcommand\Sem[1]{\left\llbracket \IMode{#1}\right\rrbracket}
\newcommand\Force[2]{\IMode{#1}\Vdash #2}
\newcommand\Val[3]{\mathcal{V}_{\IMode{#1}}^{\IMode{#2}}\Sem{#3}}
\newcommand\IsVal[5]{\Val{#1}{#2}{#3}\left(\IMode{#4},\IMode{#5}\right)}
\newcommand\Exp[3]{\mathcal{E}_{\IMode{#1}}^{\IMode{#2}}\Sem{#3}}
\newcommand\IsExp[5]{\Exp{#1}{#2}{#3}\left(\IMode{#4},\IMode{#5}\right)}
\newcommand\SetLit[1]{\left\{#1\right\}}
\newcommand\CompLit[2]{\SetLit{#1\mid #2}}
\newcommand\Sig[1]{\mathbf{\Psi}_{#1}}
\newcommand\SigUnit{\textsc{unit}}
\newcommand\TyUnit{\mathsf{unit}}
\newcommand\PrecEq[2]{\IMode{#1}\preceq\IMode{#2}}
\newcommand\SuccEq[2]{\IMode{#1}\succeq\IMode{#2}}
\newcommand\SigProd{\textsc{prod}}
\newcommand\TyProd[2]{#1\times #2}
\newcommand\Pair[2]{\left\langle #1, #2 \right\rangle}
\newcommand\SigFun{\textsc{fun}}
\newcommand\TyFun[2]{#1\supset #2}
\newcommand\Lam[2]{(\lambda #1)#2}
\newcommand\Member[2]{\IMode{#1} \in \IMode{#2}}
\newcommand\Eval[3]{\IMode{#2}\Downarrow_{#1}\OMode{#3}}
\newcommand\Worlds{\mathcal{W}}
\newcommand\MkWorld[3]{\left\langle #1, #2, #3\right\rangle}
\newcommand\Pow[1]{\mathbb{P}\left(#1\right)}
\newcommand\Term[1]{\mathcal{B}_{#1}}
\newcommand\SubsetEq[2]{\IMode{#1}\subseteq\IMode{#2}}
\newcommand\Dom[1]{\mathbf{D}_{#1}}
\newcommand\Types[1]{\mathbf{T}_{#1}}
\newcommand\Witnesses[1]{\mathbf{K}_{#1}}
\newcommand\WitnessesOf[2]{\Witnesses{#1}\left(#2\right)}

\begin{document}

\title{Oracles and Choice Sequences for Type-Theoretic Pragmatics}

\author{Jonathan Sterling}
\address{}

\maketitle

\section{Model Construction}

We will now ground the intuitive semantics for the extended theory in a
concrete model construction, using Kripke logical relations.  Let $\Term\Psi$
be the set of abstract binding trees (terms) generated by the signature $\Psi$.
Then, we define a world $w$ as a triple
$\MkWorld{\Sig{w}}{\Types{w}}{\Witnesses{w}}$, where $\Types{w}$ is the
collection of terms which have been recognized as canonical types so far, and
$\WitnessesOf{w}{A}$ is the collection of witnesses of the truth of $A$ that have
been constructed so far, for each $A$ in $\Types{w}$. In other words, the set
of worlds $\Worlds$ is defined as follows:

\[
  \Worlds
    \triangleq
     \coprod_{\Psi\in \mathsf{sig}}
      \coprod_{T\in\Pow{\Term{\Psi}}}
       T\to\Pow{\Term\Psi}
\]

Let $\Dom{w}$ be the domain of discourse $\Term{\Psi{w}}$. The accessibility
relation $\preceq$ is defined as follows:
\begin{align*}
  \PrecEq{u}{v}
   &\triangleq
    \PrecEq{\Sig{u}}{\Sig{v}}\\
     &\mathrel\land
      \SubsetEq{\Types{u}}{\Types{v}}\\
       &\mathrel\land
        \forall\Member{A}{\Types{u}}.\ \SubsetEq{\WitnessesOf{u}{A}}{\WitnessesOf{v}{A}}
\end{align*}

\begin{thm}
  $\Pair{\Worlds}{\preceq}$ is a Kripke frame.
\end{thm}
\begin{proof}
  It suffices to show that $\preceq$ is reflexive and transitive, which is immediate from the corresponding properties of signature subsumption and subsethood.
\end{proof}

\begin{align*}
  \SigUnit
    &\triangleq
    \cdot, \TyUnit:(), \bullet:()\\
  \SigProd
    &\triangleq
    \cdot, \TyProd{}{}:(0;0), \Pair{-}{-}:(0;0)\\
  \SigFun
    &\triangleq
    \cdot, \TyFun{}{}:(0;0), \lambda:(1)\\
\end{align*}

\begin{align*}
  \Val{w}{\alpha}{\TyUnit}
    &\triangleq
    \CompLit{(\bullet, \bullet)}{
      \SuccEq{\Sig{w}}{\SigUnit}
    }\\
  \Val{w}{\alpha}{\TyProd{A}{B}}
    &\triangleq
    \CompLit{
      (\Pair{M}{N} , \Pair{M'}{N'})
    }{
      \SuccEq{\Sig{w}}{\SigProd}
      \mathrel\land
      \IsExp{w}{\alpha}{A}{M}{M'}
      \mathrel\land
      \IsExp{w}{\alpha}{A}{N}{N'}
    }\\
  \Val{w}{\alpha}{\TyFun{A}{B}}
    &\triangleq
    \{
      (\Lam{x}{E}, \Lam{y}{F})
    \mid
      \SuccEq{\Sig{w}}{\SigFun}
      \mathrel\land
      \forall\SuccEq{u}{w}.
      \forall\Member{M,N}{\Dom{u}}.\\
      & \phantom{xxxx}\IsVal{u}{\alpha}{A}{M}{N} \Rightarrow
        \IsExp{u}{\alpha}{B}{[M/x]E}{[N/y]F}
    \}\\
  \Exp{w}{\alpha}{A}
    &\triangleq
    \CompLit{
      (M,N)
    }{
      \Force{w}{\Eval{\alpha}{M}{M'}}
      \mathrel\land
      \Force{w}{\Eval{\alpha}{N}{N'}}
      \mathrel\land
      \IsVal{w}{\alpha}{A}{M'}{N'}
    }
\end{align*}

\bigskip

\end{document}
