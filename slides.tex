\documentclass[10pt]{beamer}

\usefonttheme{professionalfonts}% use own font handling
\usetheme{m}

\usepackage{ulem}
\usepackage{turnstile}
\usepackage{proof}
\usepackage{booktabs}
\usepackage[scale=2]{ccicons}
\usepackage{amsmath, stmaryrd, scalerel}
\usepackage{fontspec}
\usepackage{geometry}
\usepackage{proof}
\usepackage{supertabular}
\usepackage{unicode-math}
\usepackage{xltxtra}
\usepackage{xunicode}
\setmathfont{Asana Math}

% \theoremstyle{mystyle}
\newtheorem*{remark}{Remark}

% colors
\definecolor{MidnightBlue}{rgb}{0.1, 0.1, 0.44}
\definecolor{Maroon}{rgb}{0.5, 0.0, 0.0}
\def\InputModeColorName{MidnightBlue}
\def\OutputModeColorName{Maroon}

% modes
\newcommand\IMode[1]{{\color{\InputModeColorName}{#1}}}
\newcommand\OMode[1]{{\color{\OutputModeColorName}{#1}}}

% judgments
\newcommand\GenJ[2]{%
  \vert_{\IMode{#1}}\; #2
}
\newcommand\HypJ[2]{%
  #1\ \left(#2\right)
}

\newcommand\SemBrackets[1]{\left\llbracket #1\right\rrbracket}
\newcommand\Meaning[1]{\SemBrackets{\textit{#1}}}
\newcommand\VertSem[2]{%
  \begin{array}{c}%
    \Meaning{\IMode{#1}}\\%
    \bigtriangledown\\%
    \OMode{#2}%
  \end{array}%
}

\newcommand\Force[2]{\IMode{#1}\Vdash #2}
\newcommand\ForceAlign[2]{\IMode{#1}&\Vdash#2}

\newcommand\ForceND[3]{
   \IMode{#1}%
   \sdtstile{\IMode{#2}}{}%
   #3
}

\newcommand\Eval[2]{\IMode{#1}\Downarrow\OMode{#2}}
\newcommand\VertEval[2]{
  \begin{array}{c}%
    \IMode{#1}\\%
    \Downarrow\\%
    \OMode{#2}%
  \end{array}%
}
\newcommand\Seq[2]{\IMode{#1}\vdash\IMode{#2}}
\newcommand\Member[2]{\IMode{#1}\in\IMode{#2}}
\newcommand\EqMember[3]{\IMode{#1}=\IMode{#2}\in\IMode{#3}}
\newcommand\IsType[1]{\IMode{#1}\ \mathit{type}}
\newcommand\EqType[2]{\IMode{#1}=\IMode{#2}\ \mathit{type}}
\newcommand\IsTrue[1]{\IMode{#1}\ \mathit{true}}
\newcommand\OpSig[2]{\IMode{#1} : \OMode{#2}}
\newcommand\NotationDef[2]{\IMode{#1} \eqdef \OMode{#2}}
\newcommand\Match[2]{\IMode{#1}=\OMode{#2}}
\newcommand\MatchAlign[2]{\IMode{#1}&=\OMode{#2}}
\newcommand\Size[1]{\left\vert#1\right\vert}
\newcommand\IsLessThan[2]{\IMode{#1}<\IMode{#2}}

\newcommand\Model[1]{\mathcal{M}_{#1}}

% expressions
\newcommand\TyEnt{\mathbb{E}}
\newcommand\TyFun[3]{(\Pi #2\in #1)\, #3}
\newcommand\TyImp[2]{#1\supset #2}
\newcommand\TyProd[3]{(\Sigma #2\in #1)\, #3}
\newcommand\TySet[3]{\{#2: #1\mid #3\}}
\newcommand\TySetSimple[2]{\{#1\mid #2\}}
\newcommand\Proj[1]{\pi_{#1}}
\newcommand\OpRequire{\mathbf{require}}
\newcommand\Require[3]{\OpRequire\ #2:#1\ \mathbf{in}\ #3}
\newcommand\Head[1]{\mathsf{hd}(#1)}
\newcommand\Tail[1]{\mathsf{tl}(#1)}

\newcommand\BKS[1]{\varkappa_{#1}}
\newcommand\Nil{\langle\rangle}
\newcommand\Spw{\mathfrak{S}}
\newcommand\Lawful[1]{\Spw(\IMode{#1})}

\title{oracles and choice sequences for type-theoretic pragmatics}
\date{October 8, 2015}
\institute{joint work with Darryl McAdams}
\author{Jon Sterling}

\begin{document}

\maketitle

\section{Introduction}

\begin{frame}{natural language semantics and anaphora}
  \pause
  \[
    \VertSem{A woman walked in.}{
      \pause
      \TyProd{Woman}{p}{\pause WalkedIn(p)}
    }
  \]
\end{frame}

\begin{frame}{natural language semantics and anaphora}
  \[
    \VertSem{She sat down}{
      \pause
      SatDown(???)
    }
  \]
\end{frame}

\begin{frame}{natural language semantics and anaphora}
  \[
    \VertSem{A woman walked in. She sat down}{
      \pause
      \TyProd{\TyProd{Woman}{p}{WalkedIn(p)}}{x}{
        \pause
        SatDown(???)
      }
    }
  \]
\end{frame}

\begin{frame}{natural language semantics and anaphora}
  \[
    \VertSem{A woman walked in. She sat down}{
      \TyProd{\TyProd{Woman}{p}{WalkedIn(p)}}{x}{
        SatDown(\Proj{1}(x))
      }
    }
  \]
\end{frame}

\begin{frame}{the ``donkey sentence''}
  \[
    \VertSem{Every farmer who owns a donkey beats it.}{
      \TyFun{
        \TyProd{Farmer}{x}{
          \TyProd{Donkey}{y}{
            Owns(x;y)
          }
        }
      }{p}{
        \pause
        Beats(\only<2>{???;???}\only<3>{\Proj{1}(p);\Proj{1}(\Proj{2}(p))})
      }
    }
  \]
\end{frame}

\begin{frame}{two things to deal with}
  \pause
  \begin{itemize}[<+->]
    \item \alert<4-5>{terms for presuppositions}\only<5>{ (this talk)}
    \item resolution of presuppositions
  \end{itemize}
\end{frame}

\section{the $\OpRequire$ oracle: statics}

\begin{frame}{$\OpRequire$ --- formal rules}
  \pause

  \begin{gather*}
    \OpSig{\OpRequire}{(0;1)}\tag{operator}\\
    \NotationDef{\Require{A}{x}{N}}{\OpRequire(A; x.N)}\tag{notation}
  \end{gather*}
  \pause
  \[
    \infer{
      \Seq{\Gamma}{\Member{\Require{A}{x}{N}}{B}}
    }{
      \Seq{\Gamma}{\IsType{A}} &
      \Seq{\Gamma,x:A}{\Member{N}{B}}
    }\tag{\textit{require}}
  \]
\end{frame}

\begin{frame}{$\OpRequire$ --- examples}
  \[
    \VertSem{A woman walked in. She sat down}{
      \TyProd{\TyProd{Woman}{p}{WalkedIn(p)}}{x}{
        \only<1>{SatDown(???)}%
        \only<2>{\Require{Woman}{y}{SatDown(y)}}%
      }
    }
  \]
\end{frame}

\begin{frame}{revising montague's program}
  \only<1>{
    The meaning of a sentence is a logical proposition.
  }
  \only<2->{
    \sout{The meaning of a sentence is a logical proposition.}
  }

  \only<3->{
    \alert<4->{The meaning of a sentence is a type-theoretic expression which may
    \emph{evaluate} to a canonical proposition.}
  }

  \only<5->{
    \textbf{What we want:}
    \begin{gather*}
      \TyProd{\TyProd{Woman}{p}{WalkedIn(p)}}{x}{\Require{Woman}{y}{SatDown(y)}}\\
      \sim\\
      \TyProd{\TyProd{Woman}{p}{WalkedIn(p)}}{x}{\Proj{1}(x)}
    \end{gather*}
  }

  \only<6->{
    \centering
    \framebox{
      where
      $\IMode{M\sim N} \eqdef
       \OMode{(M\preceq N)\land(N\preceq M)}$
    }
  }
\end{frame}

\begin{frame}{Every grammatical sentence has a meaning}
  \pause
  ...but only some of them denote propositions (types)!
\end{frame}

\begin{frame}{a negative example}
  \pause
  \[
    \VertSem{
      The unicorn ran a marathon
    }{
      \pause
      \Require{Unicorn}{x}{
        \TyProd{Marathon}{y}{
          Ran(x;y)
        }
      }
    }
  \]

  \pause
  \begin{center}
    (not a proposition)
  \end{center}
\end{frame}

\begin{frame}{a positive example}
  \pause
  \[
    \VertSem{
      The President ran a marathon
    }{
      \pause
      \Require{President}{x}{
        \TyProd{Marathon}{y}{
          Ran(x;y)
        }
      }
    }
  \]
\end{frame}

\begin{frame}{a positive example}
  \pause
  \[
    \VertEval{
      \Require{President}{x}{
        \TyProd{Marathon}{y}{
          Ran(x;y)
        }
      }\pause
    }{
      \TyProd{Marathon}{y}{
        Ran(Obama;y)
      }
    }
  \]

  \pause
  \begin{center}
    Evaluation is now non-deterministic
  \end{center}
\end{frame}

\plain{Is $\OpRequire$ computationally effective?}

\begin{frame}
  \emph{Yes, but we need two things:}\pause
  \begin{enumerate}[<+->]
    \item knowledge-sensitive judgments \only<4->{\alert{(forcing)}}
    \item \only<2-6>{non-}deterministic computation \only<6->{\alert{(use choice sequences)}}
  \end{enumerate}

  \begin{center}
    \only<5-7>{
      \framebox{
        $\Force{w}{\mathcal{J}}$
      }
    }
    \only<8->{
      \framebox{
        $\ForceND{\alpha}{w}{\mathcal{J}}$
      }
    }
  \end{center}
\end{frame}

\section{assertion acts in time}

\begin{frame}{beth-kripke semantics for assertions}
  assertion acts (judgments) are intensional (local)
  \pause

  \begin{align*}
    \only<8->{\IMode{w}}&\only<8->{\Vdash}\only<2->{\GenJ{x}{\mathcal{J}(x)}\tag{general judgment}}\\
    \only<8->{\IMode{w}}&\only<8->{\Vdash}\only<3->{\HypJ{\mathcal{J}_2}{\mathcal{J}_1}\tag{hypothetical judgment}}\\
    \only<8->{\IMode{w}}&\only<8->{\Vdash}\only<4->{\Eval{M}{N}\tag{evaluation}}\\
    \only<8->{\IMode{w}}&\only<8->{\Vdash}\only<5->{\EqType{A}{B}\tag{typehood}}\\
    \only<8->{\IMode{w}}&\only<8->{\Vdash}\only<6->{\IsTrue{A}\tag{truth}}\\
    \only<8->{\IMode{w}}&\only<8->{\Vdash}\only<7->{\EqMember{M}{N}{A}\tag{membership}}
  \end{align*}
\end{frame}

\begin{frame}{semantics of higher-order assertions}
  \only<1>{
    \begin{align*}
      \ForceAlign{w}{\GenJ{x}{\mathcal{J}(x)}}\tag{general judgment}\\
      \ForceAlign{w}{ \HypJ{\mathcal{J}_2}{\mathcal{J}_1}}\tag{hypothetical judgment}\\
    \end{align*}
  }
  \only<2->{
    \begin{alignat*}{3}
      \ForceAlign{w}{\GenJ{x}{\mathcal{J}(x)}}%
        &\Longleftrightarrow \ \ \ &
        \only<2>{\OMode{\cdots}}
        \only<3->{\forall \IMode{u}\succeq\IMode{w}.\forall\Member{x}{\Model{u}}.\ \IMode{u}\Vdash\mathcal{J}(x)}\\
      \IMode{w}&\Vdash\HypJ{\mathcal{J}_2}{\mathcal{J}_1}\ \ \ %
         &\Longleftrightarrow \ \ \ &
         \only<2>{\OMode{\cdots}}
         \only<3->{\forall \IMode{u}\succeq\IMode{w}.\ \IMode{u}\Vdash\mathcal{J}_1\Rightarrow\IMode{u}\Vdash\mathcal{J}_2}
    \end{alignat*}
  }
  \only<4->{%
    where $\Model{w}$ is the species of constructions that have been effected by stage $w$
  }
\end{frame}

\begin{frame}{intensional / ephemeral truth (kripke)}
  \[
    \IMode{w}\Vdash\IsTrue{A}
      \Longleftrightarrow\pause
       \exists\Member{m}{\Model{w}}.\ \pause
        \IMode{w}\Vdash\EqMember{m}{m}{A}
  \]
\end{frame}

\begin{frame}{intensional / ephemeral truth (beth)}
  \[
    \IMode{w}\Vdash\IsTrue{A}
      \Longleftrightarrow\pause
       \exists\IMode{\mathfrak{B}}\;\mathbf{bars}\;\IMode{w}.\pause
        \forall\Member{u}{\mathfrak{B}}.\pause
         \exists\Member{m}{\Model{u}}.\ \pause
          \IMode{u}\Vdash\EqMember{m}{m}{A}
  \]
\end{frame}

%%% TODO: introduce creating subject

\section{the $\OpRequire$ oracle: dynamics}

\begin{frame}{plan of action}
  We need to explain what it means for the following judgment to be forced:
  \[
    \Eval{\Require{A}{x}{N}}{N'}
  \]
  \pause

  The collection of possible reductions is provided via the \alert{theory of the Creating Subject}.\pause

  Non-determinism may be eliminated via by stating the judgments of type theory \alert{relative to a choice sequence}, subject to a spread law.
\end{frame}

\begin{frame}{the theory of the creating subject}
  \pause
  All mathematics is a mental construction in the mind of an idealized subject, subject to the following laws:
  \pause
  \begin{enumerate}[<+->]
    \item experiences are never forgotten \alert{(monotonicity, functoriality)}
    \item at a point in time, the subject knows whether or not it has experienced a judgment \alert{(decidability)}
      \begin{remark}
        Contra Dummett, I \emph{by no means} take the above as requiring that the following shall be true in a constructive metatheory, \alert{divorced from time}:
        \[
          \forall\IMode{w}.\forall\IMode{\mathcal{J}}.\ \SemBrackets{\Force{w}{\IMode{\mathcal{J}}}} \lor \lnot \SemBrackets{\Force{w}{\IMode{\mathcal{J}}}}\tag{infelicity}
        \]
        The above is impossible in a Beth model.
      \end{remark}
  \end{enumerate}

\end{frame}

\begin{frame}{$\OpRequire$ --- dynamics}
  \begin{gather*}
    \infer{
      \Lawful{\Nil}
    }{
    }
    \qquad
    \infer{
      \Lawful{\vec{u}\frown\rho}
    }{
      \Lawful{\vec{u}} &
      \GenJ{n}{
        \HypJ{
          \IsLessThan{\mathbb{\rho}(n)}{n}
        }{
          \Member{n}{\mathbb{N}^+}
        }
      }
    }\tag{spread law}
  \end{gather*}

  \begin{gather*}
    \infer{
      \ForceND{\alpha}{t}{
        \Eval{\Require{A}{x}{N}}{N'}
      }
    }{
      \ForceND{\alpha}{t}{\Eval{A}{A'}} &
      \Match{\Size{\BKS{A'}(t)}}{\ell} &
      \Match{\Head{\alpha}(\ell)}{j} &
      \ForceND{\Tail{\alpha}}{t}{
        \Eval{\left[\BKS{A'}(j)/x\right]N}{N'}
      }
    }\tag{for $\Member{\alpha}{\Spw}$}
  \end{gather*}

\end{frame}

\plain{Questions?}

%%\begin{frame}[allowframebreaks]
%%
%%  \frametitle{References}
%%
%%  \bibliography{refs}
%%  \bibliographystyle{abbrv}
%%
%%\end{frame}

\end{document}
